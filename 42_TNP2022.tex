\documentclass[12pt]{article}

\usepackage{graphicx}
\usepackage{cmsrb}
\usepackage[OT2,T1]{fontenc} %better to use T1, but OT1 will also work
\usepackage[serbian]{babel}
\usepackage[utf8]{inputenc}
\usepackage[a4paper,top=2cm,bottom=2cm,left=3cm,right=3cm,marginparwidth=1.75cm]{geometry}
\usepackage{times}

	

\usepackage[colorlinks=true, allcolors=blue]{hyperref}

\begin{document}
\title{ИТ конференције у Србији 2022. године\\ \small{Seminarski rad u okviru kursa\\Tehničko i naučno pisanje\\ Matematički fakultet}}
\author{Невена Јокић\\ nevenajokic2003@gmail.com \and Андреј Михајлов\\ andrejm17@gmail.com \and Душан Лукић\\ mejl \and Јован Гагић\\ jovangagiccofi@gmail.com}
\date{23.~avgust 2023.}
\maketitle
\abstract{Овај рад анализира улогу и допринос конференција, посебно ИТ конференција, у данашњем друштву у Србији и у свету.}

\newpage

\tableofcontents
\newpage

\section{Увод}

Конференције су планирани догађаји на којима се професионалци или стручњаци окупљају како би поделили знање, разменили идеје и повезали се са другима у истој области. Ове прилике се често дешавају у различитим областима и секторима, као што су технологија, наука, медицина, бизнис, образовање, уметност и другe. Појам конферфенција оригинално потиче из латинског језика од речи \textit{conferentia} што значи састанак или окупљање. За сазивање конференције није потребна никаква традиција, континуитет или периодичност конференције. Иако углавном нису временски ограничене, конференције су обично кратког трајања
са специфичним циљевима. Због сличности у именима долази до мешања са конвенцијама, али конвенције су у већини случајева доста већи догађаји који се састоје од делегација које представљају различите групе.
Скупови, састанци и догађаји били су део живота људи од најраније забележене историје. Археолози су пронашли примитивне рушевине од давнина културе које су се користиле као простори за састанке где би се грађани окупљали да разговарају о заједничким интересима, као што су влада, рат, лов или племенске прославе. Једном када људи развију отворена стална насеља, сваки град или село је имао јавни простор за састанке, који се често звао градски трг, где су се становници могли састати, разговарати и прославити. Прва позната употеба речи конфереција се појављује у 16. веку, међутим идеја о конференцији се појавила далеко пре употребе. 
Конференције могу имати различите формате, теме и намере. Што се тиче формата конференције разликујемо следеће:
\begin{enumerate}
    \item Конференцијски позив – у телекомуникацијама, представља позив од два или више учесника у исто време. Први забележен конференцијски позив oбавио је 1915. године Александар Бел, први власник патента за телефон у Сједињеним Америчким Државама. Његов помоћник Вотсон, као и градоначелници Њујорка и Сан Франциска, придружили су се Белу на првом конференцијском позиву 1915.
    \item Конференцијске сале – сале и просторије у којима се одржавају конференције.
    \item Видео конференције – конференције које се одвијају преко аудио-видео позива са учесницима на различитим локацијама. Током пандемије вируса ковид-19, видео конференције су постале веома популарне и неопходне да би нека конференција могла да се одржи. Међутим, њихова популарност се одржала и наставила да расте чак и након проглашења завршетка пандемије.
\end{enumerate}
На конференцијама се расправља о иновативним идејама и размењују се нове информације међу стручњацима. По намери конференција разликујемо следеће:
\begin{enumerate}
    \item Пословне конференције – конференције које се одржавају за исту компанију или индустрију. Стручњаци се окупљају да би дискутовали о новим приликама које се тичу њиховог посла.
    \item Академске конференције – конференције на којима се окупљају научници или академици који презентују своја истраживања о којима се затим дискутује.
    \item Трговинска конференција – конференције већег обима на којима се поред бизнисмена и привредника скупљају и грађани који долазе да се повежу са продавцима и остваре нове везе.
    \item Некоференција – конференција која избегава велике трошкове и спонзорисане презентације. Сви учесници су једнако упућени у тему конференције и дискусија је отвореног типа тојест некоференције углавном немају једног главног говорника који се обраћа скупу.
\end{enumerate}

\pagebreak

\section{ИТ конференције}
Информационе технологије или скраћено ИТ представљају дисциплину која подразумева коришћење рачунара за складишћење, анализу, преузимање, пренос и манипулацију података или информација. Термин информационе технологије се користи у много ширем смислу који подразумева све активности којима се ИТ стручњаци баве, од инсталација апликативних програма до пројектовања сложених рачунарских мрежа.
ИТ конференције су конференције које за циљ имају да окупе стручњаке и истраживаче из поља информационих технологија као и људе који желе да науче више о њима. Учесници се окупљају да дискусују о новим идејама, трендовима и развоју ИТ сектора. Постоје стотине релевантних подтема у овој области као што су сајбер безбедност, виртуелна реалност, вештачка интелигенција и машинско учење, технолошко планирање за унапређење образовања, дигиталне технологије и још много тога. Величина и фокус конференција о информационим технологијама могу варирати. Неке се дешавају лично, док су неке виртуелне. ИТ конференције учесницима дају могућност умрежавања са другим професионалцима како би унапредили своју каријеру и пословање као и искуство из прве руке са најновијим алатима и технологијама у овој области.
\subsection{ИТ конференције у Србији}
Србија, као земља која све више препознаје потенцијал информационих технологија, постала је и 2022. године домаћин разноврсним и великим ИТ конференцијама које су окупиле стручњаке и знатижељне из целог света. Међу огромним бројем конференција које су се одржале те године највише су се истакле: Синергија, “COMING IT” конференција, TMRW Belgrade конференција, OpenIt, Telfor konferencija, Data Science konferencija.

\end{document}